\PassOptionsToPackage{unicode=true}{hyperref} % options for packages loaded elsewhere
\PassOptionsToPackage{hyphens}{url}
%
\documentclass[]{article}
\usepackage{lmodern}
\usepackage{amssymb,amsmath}
\usepackage{ifxetex,ifluatex}
\usepackage{fixltx2e} % provides \textsubscript
\ifnum 0\ifxetex 1\fi\ifluatex 1\fi=0 % if pdftex
  \usepackage[T1]{fontenc}
  \usepackage[utf8]{inputenc}
  \usepackage{textcomp} % provides euro and other symbols
\else % if luatex or xelatex
  \usepackage{unicode-math}
  \defaultfontfeatures{Ligatures=TeX,Scale=MatchLowercase}
\fi
% use upquote if available, for straight quotes in verbatim environments
\IfFileExists{upquote.sty}{\usepackage{upquote}}{}
% use microtype if available
\IfFileExists{microtype.sty}{%
\usepackage[]{microtype}
\UseMicrotypeSet[protrusion]{basicmath} % disable protrusion for tt fonts
}{}
\IfFileExists{parskip.sty}{%
\usepackage{parskip}
}{% else
\setlength{\parindent}{0pt}
\setlength{\parskip}{6pt plus 2pt minus 1pt}
}
\usepackage{hyperref}
\hypersetup{
            pdfborder={0 0 0},
            breaklinks=true}
\urlstyle{same}  % don't use monospace font for urls
\usepackage{color}
\usepackage{fancyvrb}
\newcommand{\VerbBar}{|}
\newcommand{\VERB}{\Verb[commandchars=\\\{\}]}
\DefineVerbatimEnvironment{Highlighting}{Verbatim}{commandchars=\\\{\}}
% Add ',fontsize=\small' for more characters per line
\newenvironment{Shaded}{}{}
\newcommand{\AlertTok}[1]{\textcolor[rgb]{1.00,0.00,0.00}{\textbf{#1}}}
\newcommand{\AnnotationTok}[1]{\textcolor[rgb]{0.38,0.63,0.69}{\textbf{\textit{#1}}}}
\newcommand{\AttributeTok}[1]{\textcolor[rgb]{0.49,0.56,0.16}{#1}}
\newcommand{\BaseNTok}[1]{\textcolor[rgb]{0.25,0.63,0.44}{#1}}
\newcommand{\BuiltInTok}[1]{#1}
\newcommand{\CharTok}[1]{\textcolor[rgb]{0.25,0.44,0.63}{#1}}
\newcommand{\CommentTok}[1]{\textcolor[rgb]{0.38,0.63,0.69}{\textit{#1}}}
\newcommand{\CommentVarTok}[1]{\textcolor[rgb]{0.38,0.63,0.69}{\textbf{\textit{#1}}}}
\newcommand{\ConstantTok}[1]{\textcolor[rgb]{0.53,0.00,0.00}{#1}}
\newcommand{\ControlFlowTok}[1]{\textcolor[rgb]{0.00,0.44,0.13}{\textbf{#1}}}
\newcommand{\DataTypeTok}[1]{\textcolor[rgb]{0.56,0.13,0.00}{#1}}
\newcommand{\DecValTok}[1]{\textcolor[rgb]{0.25,0.63,0.44}{#1}}
\newcommand{\DocumentationTok}[1]{\textcolor[rgb]{0.73,0.13,0.13}{\textit{#1}}}
\newcommand{\ErrorTok}[1]{\textcolor[rgb]{1.00,0.00,0.00}{\textbf{#1}}}
\newcommand{\ExtensionTok}[1]{#1}
\newcommand{\FloatTok}[1]{\textcolor[rgb]{0.25,0.63,0.44}{#1}}
\newcommand{\FunctionTok}[1]{\textcolor[rgb]{0.02,0.16,0.49}{#1}}
\newcommand{\ImportTok}[1]{#1}
\newcommand{\InformationTok}[1]{\textcolor[rgb]{0.38,0.63,0.69}{\textbf{\textit{#1}}}}
\newcommand{\KeywordTok}[1]{\textcolor[rgb]{0.00,0.44,0.13}{\textbf{#1}}}
\newcommand{\NormalTok}[1]{#1}
\newcommand{\OperatorTok}[1]{\textcolor[rgb]{0.40,0.40,0.40}{#1}}
\newcommand{\OtherTok}[1]{\textcolor[rgb]{0.00,0.44,0.13}{#1}}
\newcommand{\PreprocessorTok}[1]{\textcolor[rgb]{0.74,0.48,0.00}{#1}}
\newcommand{\RegionMarkerTok}[1]{#1}
\newcommand{\SpecialCharTok}[1]{\textcolor[rgb]{0.25,0.44,0.63}{#1}}
\newcommand{\SpecialStringTok}[1]{\textcolor[rgb]{0.73,0.40,0.53}{#1}}
\newcommand{\StringTok}[1]{\textcolor[rgb]{0.25,0.44,0.63}{#1}}
\newcommand{\VariableTok}[1]{\textcolor[rgb]{0.10,0.09,0.49}{#1}}
\newcommand{\VerbatimStringTok}[1]{\textcolor[rgb]{0.25,0.44,0.63}{#1}}
\newcommand{\WarningTok}[1]{\textcolor[rgb]{0.38,0.63,0.69}{\textbf{\textit{#1}}}}
\usepackage{longtable,booktabs}
% Fix footnotes in tables (requires footnote package)
\IfFileExists{footnote.sty}{\usepackage{footnote}\makesavenoteenv{longtable}}{}
\setlength{\emergencystretch}{3em}  % prevent overfull lines
\providecommand{\tightlist}{%
  \setlength{\itemsep}{0pt}\setlength{\parskip}{0pt}}
\setcounter{secnumdepth}{0}
% Redefines (sub)paragraphs to behave more like sections
\ifx\paragraph\undefined\else
\let\oldparagraph\paragraph
\renewcommand{\paragraph}[1]{\oldparagraph{#1}\mbox{}}
\fi
\ifx\subparagraph\undefined\else
\let\oldsubparagraph\subparagraph
\renewcommand{\subparagraph}[1]{\oldsubparagraph{#1}\mbox{}}
\fi

% set default figure placement to htbp
\makeatletter
\def\fps@figure{htbp}
\makeatother


\date{}

\begin{document}

\hypertarget{ux63a5ux53e3ux6587ux6863}{%
\section{接口文档}\label{ux63a5ux53e3ux6587ux6863}}

\hypertarget{ux57faux672cux7ea6ux5b9a}{%
\subsection{基本约定}\label{ux57faux672cux7ea6ux5b9a}}

\begin{enumerate}
\def\labelenumi{\arabic{enumi}.}
\tightlist
\item
  返回值格式是JSON \{err:0, msg:'', result:
  Object\},0为没有错误,操作成功,非0有msg详细信息。 result
  的键值和具体请求有关,如果 result 有一个 key 是
  \texttt{student\_list},那么有:
\end{enumerate}

\begin{Shaded}
\begin{Highlighting}[]
\NormalTok{result[}\StringTok{'student_list'}\NormalTok{] }\OperatorTok{=}\NormalTok{ [[name}\OperatorTok{,}\NormalTok{ school]}\OperatorTok{,}\NormalTok{[name_2}\OperatorTok{,}\NormalTok{ school_2]]}
\end{Highlighting}
\end{Shaded}

\begin{enumerate}
\def\labelenumi{\arabic{enumi}.}
\setcounter{enumi}{1}
\tightlist
\item
  学校名称与代号对照表
\end{enumerate}

\begin{longtable}[]{@{}ll@{}}
\toprule
school & short\_code\tabularnewline
\midrule
\endhead
哈工大 & hit\tabularnewline
深大 & szu\tabularnewline
清华 & thu\tabularnewline
北大 & pku\tabularnewline
南科大 & sust\tabularnewline
\bottomrule
\end{longtable}

\begin{enumerate}
\def\labelenumi{\arabic{enumi}.}
\setcounter{enumi}{2}
\item
  \texttt{\$root} 表示应用部署的根URL。
\item
  semester 这个参数为1表示2018年秋季学期,为2表示2019年春季学期,在 GET
  和 POST 请求中,不提供这个参数默认为2。
\item
  小组名称(2019年春季学期):周一下午、周二下午、周四下午金色年华、周四下午童伴时光、周五下午单周、周四下午双周
  \#\# Web API
\item
  添加1个学生到2019年春季学期的流动组
\end{enumerate}

\begin{verbatim}
curl -X POST $root/xingyu/add_student_flow.php -H "Content-Type: application/json" -d '{"student_name":"张三","student_school":"hit"}'
\end{verbatim}

必有参数为 student\_name(string),
student\_school(enum),school只能从五个学校的代号里选取。 返回结果 err =
3 时表示该学生已经存在。

\begin{enumerate}
\def\labelenumi{\arabic{enumi}.}
\setcounter{enumi}{1}
\tightlist
\item
  根据姓名的前几个汉字获取流动组学生列表不分页
\end{enumerate}

\begin{verbatim}
curl -X GET "$root/xingyu/get_student_list.php?student_name_prefix=张&semester=2"
\end{verbatim}

返回结果为
\texttt{result{[}\textquotesingle{}student\_list\textquotesingle{}{]}}

\begin{enumerate}
\def\labelenumi{\arabic{enumi}.}
\setcounter{enumi}{2}
\tightlist
\item
  获取某个活动的固定志愿者
\end{enumerate}

\begin{verbatim}
curl -X GET $root/xingyu/get_fixed_student.php?student_group=[group_name]&semester=2
\end{verbatim}

返回结果为
\texttt{result{[}\textquotesingle{}student\_list\textquotesingle{}{]}}

\begin{enumerate}
\def\labelenumi{\arabic{enumi}.}
\setcounter{enumi}{3}
\tightlist
\item
  批量添加某个活动参与的全部学生
\end{enumerate}

\begin{verbatim}
curl -X POST $root/xingyu/add_activity.php -H "Content-Type: application/json" -d '{"week":3,"name":"周二下午", "student_list":["张三"]}'
\end{verbatim}

必有参数为 深大的周数(week),int;{[}3-18{]}
和小组名称(name)(string);以及学生名字列表(student\_list:{[}`name\_1',`name\_2'{]})

返回结果 err = 5 时表示该活动已经存在。

\begin{enumerate}
\def\labelenumi{\arabic{enumi}.}
\setcounter{enumi}{4}
\tightlist
\item
  获取某学期小组列表
\end{enumerate}

\begin{verbatim}
curl -X GET $root/xingyu/get_group_list.php&semester=2
\end{verbatim}

返回结果为
\texttt{result{[}\textquotesingle{}group\_list\textquotesingle{}{]}\ =\ {[}{[}group\_name\_1{]},{[}group\_name\_2{]}{]}}

\begin{enumerate}
\def\labelenumi{\arabic{enumi}.}
\setcounter{enumi}{5}
\tightlist
\item
  获取某个活动的全部志愿者(用于提交成功后查看结果)
\end{enumerate}

\begin{verbatim}
curl -X GET "$root/xingyu/get_all_student.php?student_group=周二下午&week=4" # 常规活动请求格式
curl -X GET "$root/xingyu/get_all_student.php?name=前期体验活动&location=金色年华&time=2019-03-05" # 常规和扩展活动请求格式
\end{verbatim}

返回结果为
\texttt{result{[}\textquotesingle{}student\_list\textquotesingle{}{]}}

\begin{enumerate}
\def\labelenumi{\arabic{enumi}.}
\setcounter{enumi}{6}
\tightlist
\item
  补录某个活动参与的学生
\end{enumerate}

\begin{verbatim}
curl -X POST $root/xingyu/append_activity.php -H "Content-Type: application/json" -d '{"week":4,"name":"周二下午",student_list":["张三"]}'
\end{verbatim}

必有参数为 深大的周数(week),int;{[}3-18{]}
和小组名称(name)(string);以及学生名字列表(student\_list:{[}`name\_1',`name\_2'{]}),其中学生名字的列表为要
补录的同学,允许同学之前已经存在。

返回结果 err = 5 时表示该活动不存在。

\begin{enumerate}
\def\labelenumi{\arabic{enumi}.}
\setcounter{enumi}{7}
\tightlist
\item
  更改1个流动组学生的学校信息
\end{enumerate}

\begin{verbatim}
curl -X POST $root/xingyu/modify_student_flow.php -H "Content-Type: application/json" -d '{"student_name":"张三","student_school":"hit"}'
\end{verbatim}

必有参数为 student\_name(string), student\_school(enum),school
只能从五个学校的代号里选取。

\begin{enumerate}
\def\labelenumi{\arabic{enumi}.}
\setcounter{enumi}{8}
\tightlist
\item
  获取所有的特色活动的信息
\end{enumerate}

\begin{verbatim}
curl -X GET $root/xingyu/get_special_activity.php
\end{verbatim}

返回的结果 result{[}`special\_activity\_list'{]} 是一个 array, 每个
array 的 item 长度为3, 依次为 name, location, time。

\begin{enumerate}
\def\labelenumi{\arabic{enumi}.}
\setcounter{enumi}{9}
\tightlist
\item
  删除某个活动的学生
\end{enumerate}

\begin{verbatim}
curl -X POST $root/xingyu/remote_activity_student.php -H "Content-Type: application/json" -d '{"week":4,"name":"周二下午",student_list":["张三"]}'
\end{verbatim}

必有参数为 深大的周数(week),int;{[}3-18{]}
和小组名称(name)(string);以及学生名字列表(student\_list:{[}`name\_1',`name\_2'{]}),其中学生名字的列表为要
删除的同学,允许同学之前没参加过该活动,此时也不会报错。 返回结果 err =
5 时表示该活动不存在。

\begin{enumerate}
\def\labelenumi{\arabic{enumi}.}
\setcounter{enumi}{10}
\tightlist
\item
  更改志愿者的组别信息
\end{enumerate}

\begin{verbatim}
curl -X POST $root/xingyu/modify_student_group.php?action=add -H "Content-Type: application/json" -d '{"student_name":"张三","group_id":2}'
\end{verbatim}

如果 \texttt{action=add} 是添加一个新的组别,如果 \texttt{action=delete}
是删除这个组别(目前没有约束)。 必有参数为 student\_name(string),
group\_id(int),要求 group\_id \textgreater{} 0.
\texttt{action=delete}情况下,返回结果 err = 5 时表示该学生不属于 id
为传递的 group\_id 的组。

\begin{enumerate}
\def\labelenumi{\arabic{enumi}.}
\setcounter{enumi}{11}
\tightlist
\item
  获取五校统计信息
\end{enumerate}

\begin{verbatim}
curl -X GET $root/xingyu/get_statistics.php
\end{verbatim}

返回 JSON格式数据
{[}\{`school':`hit',`total\_student':23,`total\_count':45\}, \ldots{}{]}
其中 total\_student 表示该校人数信息, total\_count 表示该校人次信息。

\begin{enumerate}
\def\labelenumi{\arabic{enumi}.}
\setcounter{enumi}{12}
\tightlist
\item
  删除流动志愿者
\end{enumerate}

\begin{verbatim}
curl -X POST $root/xingyu/delete_student_flow.php -H "Content-Type: application/json" -d '{"student_name":"张三","student_school":"hit"}'
\end{verbatim}

必有参数为 student\_name(string),
student\_school(enum),school只能从五个学校的代号里选取。 返回结果 err =
4 时表示该学生参与过活动,无法删除。
注意:如果这个志愿者有参加过活动,则必须先通过其他的接口取消他参加的活动才能删除。

\begin{enumerate}
\def\labelenumi{\arabic{enumi}.}
\setcounter{enumi}{13}
\tightlist
\item
  获取各校志愿者本学期的统计信息
\end{enumerate}

\begin{verbatim}
curl -X GET $root/xingyu/download_summary.php?student_school=hit
\end{verbatim}

返回 excel 格式数据,如果学校不在代号列表里面,返回的 body 为空。

\hypertarget{ux767bux5f55ux76f8ux5173}{%
\subsection{登录相关}\label{ux767bux5f55ux76f8ux5173}}

登录相关:第一次登录需同时完成1和2两步,假如只完成一步也算第一次登录失败。后面登录只需完成第1步。
1. 凭证校验

\begin{verbatim}
curl -X GET $root/xingyu/openid.php?code=abc
\end{verbatim}

其中 code 是通过 \texttt{wx.login} 从微信开发者服务器上获得的; 返回
JSON 格式的数据: 如果自有服务器请求微信开发者服务器失败,返回
\{``err'':1,``msg'':``invalid code'',``result'':"``\};
如果请求成功,返回
\{''err``:0,''msg``:''``,''result``:\{''openid``:''id``,''session\_key``:''key"\}\}

\begin{enumerate}
\def\labelenumi{\arabic{enumi}.}
\setcounter{enumi}{1}
\tightlist
\item
  关联志愿者信息与openid。
\end{enumerate}

\begin{verbatim}
curl -X POST $root/xingyu/openid.php -H "Content-Type: application/json" -d '{"openid":"abc","nickname":"张小三"}'
\end{verbatim}

返回 JSON 格式的数据,\{``err'':0,``msg'':"``,''result``:''"\},其中 err
为 0 表示操作成功。

\hypertarget{ux6743ux9650ux76f8ux5173}{%
\subsection{权限相关}\label{ux6743ux9650ux76f8ux5173}}

数据库中 student 表 里面 openid 非空者为管理员,有相应 POST 接口的权限。
所有 POST 请求 openid
是必有参数,可以置空或不填,但这种情况下肯定无法进行数据库的写操作。
如果在POST请求中返回的错误码 err =
44,说明当前用户没有权限执行这个操作。

\end{document}
