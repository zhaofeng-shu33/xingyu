\PassOptionsToPackage{unicode=true}{hyperref} % options for packages loaded elsewhere
\PassOptionsToPackage{hyphens}{url}
%
\documentclass[]{article}
\usepackage{lmodern}
\usepackage{amssymb,amsmath}
\usepackage{ifxetex,ifluatex}
\usepackage{fixltx2e} % provides \textsubscript
\ifnum 0\ifxetex 1\fi\ifluatex 1\fi=0 % if pdftex
  \usepackage[T1]{fontenc}
  \usepackage[utf8]{inputenc}
  \usepackage{textcomp} % provides euro and other symbols
\else % if luatex or xelatex
  \usepackage{unicode-math}
  \defaultfontfeatures{Ligatures=TeX,Scale=MatchLowercase}
\fi
% use upquote if available, for straight quotes in verbatim environments
\IfFileExists{upquote.sty}{\usepackage{upquote}}{}
% use microtype if available
\IfFileExists{microtype.sty}{%
\usepackage[]{microtype}
\UseMicrotypeSet[protrusion]{basicmath} % disable protrusion for tt fonts
}{}
\IfFileExists{parskip.sty}{%
\usepackage{parskip}
}{% else
\setlength{\parindent}{0pt}
\setlength{\parskip}{6pt plus 2pt minus 1pt}
}
\usepackage{hyperref}
\hypersetup{
            pdfborder={0 0 0},
            breaklinks=true}
\urlstyle{same}  % don't use monospace font for urls
\usepackage{longtable,booktabs}
% Fix footnotes in tables (requires footnote package)
\IfFileExists{footnote.sty}{\usepackage{footnote}\makesavenoteenv{longtable}}{}
\setlength{\emergencystretch}{3em}  % prevent overfull lines
\providecommand{\tightlist}{%
  \setlength{\itemsep}{0pt}\setlength{\parskip}{0pt}}
\setcounter{secnumdepth}{0}
% Redefines (sub)paragraphs to behave more like sections
\ifx\paragraph\undefined\else
\let\oldparagraph\paragraph
\renewcommand{\paragraph}[1]{\oldparagraph{#1}\mbox{}}
\fi
\ifx\subparagraph\undefined\else
\let\oldsubparagraph\subparagraph
\renewcommand{\subparagraph}[1]{\oldsubparagraph{#1}\mbox{}}
\fi

% set default figure placement to htbp
\makeatletter
\def\fps@figure{htbp}
\makeatother


\date{}

\begin{document}

\hypertarget{ux63a5ux53e3ux6587ux6863}{%
\section{接口文档}\label{ux63a5ux53e3ux6587ux6863}}

返回值格式是JSON \{err:0, msg:'', result:
Object\},0为没有错误,非0有msg详细信息。

学校代号:

\begin{longtable}[]{@{}ll@{}}
\toprule
school & short\_code\tabularnewline
\midrule
\endhead
哈工大 & hit\tabularnewline
深大 & szu\tabularnewline
清华 & thu\tabularnewline
北大 & pku\tabularnewline
南科大 & sust\tabularnewline
\bottomrule
\end{longtable}

\begin{enumerate}
\def\labelenumi{\arabic{enumi}.}
\item
  添加1个学生到2019年春季学期的流动组 POST 方法到
  /xingyu/add\_student\_flow.php (JSON格式数据)必有参数为
  student\_name(string),
  student\_school(enum),school只能从五个学校的代号里选取。 返回结果 err
  = 3 时表示该学生已经存在。
\item
  根据姓名的前几个汉字获取流动组学生列表不分页 GET 方法到
  /xingyu/get\_student\_list.php?student\_name\_prefix={[}name{]}\&semester=2
  semester=1表示2018年秋季学期,=2表示2019年春季学期,不提供这个参数默认为2。
  返回结果 : result{[}`student\_list'{]} = {[}{[}name,
  school{]},{[}name\_2, school\_2{]}{]}
\item
  获取某个活动的固定志愿者 GET 方法到
  /xingyu/get\_fixed\_student.php?student\_group={[}group\_name{]}\&semester=2
  semester=1表示2018年秋季学期,=2表示2019年春季学期,不提供这个参数默认为2。
  返回结果 : result{[}`student\_list'{]} = {[}{[}name,
  school{]},{[}name\_2, school\_2{]}{]}
\item
  批量添加某个活动参与的全部学生 POST 方法到 /xingyu/add\_activity.php
  (JSON格式数据)必有参数为 深大的周数(week),int;{[}3-18{]}
  和小组名称(name)(string);以及学生名字列表(student\_list:{[}`name\_1',`name\_2'{]})
  可选参数为semester,
  semester=1表示2018年秋季学期,=2表示2019年春季学期,不提供这个参数默认为2。
\end{enumerate}

返回结果 err = 5 时表示该活动已经存在。

\begin{enumerate}
\def\labelenumi{\arabic{enumi}.}
\setcounter{enumi}{4}
\item
  获取小组列表 GET 方法到 /xingyu/get\_group\_list.php\&semester=2
  semester=1表示2018年秋季学期,=2表示2019年春季学期,不提供这个参数默认为2。
  返回结果 : result{[}`group\_list'{]} =
  {[}{[}group\_name\_1{]},{[}group\_name\_2{]}{]}
\item
  获取某个活动的全部志愿者(用于提交成功后查看结果) GET 方法到
  /xingyu/get\_all\_student.php?student\_group={[}group\_name{]}\&week={[}week\_num{]}
  如果student\_group 和 week参数不提供,需提供 name, location 和time
  三个参数,用于查询特色活动,name 和 location 都是字符串,其中 time
  的格式是 '2019-03-05'这种。 返回结果 : result{[}`student\_list'{]} =
  {[}{[}name, school{]},{[}name\_2, school\_2{]}{]}
\item
  补录某个活动参与的学生 POST 方法到 /xingyu/append\_activity.php
  (JSON格式数据)必有参数为 深大的周数(week),int;{[}3-18{]}
  和小组名称(name)(string);以及学生名字列表(student\_list:{[}`name\_1',`name\_2'{]}),其中学生名字的列表为要
  补录的同学,允许同学之前已经存在。 可选参数为semester,
  semester=1表示2018年秋季学期,=2表示2019年春季学期,不提供这个参数默认为2。
\end{enumerate}

返回结果 err = 5 时表示该活动不存在。

\begin{enumerate}
\def\labelenumi{\arabic{enumi}.}
\setcounter{enumi}{7}
\item
  更改1个流动组学生的学校信息 POST 方法到
  /xingyu/modify\_student\_flow.php (JSON格式数据)必有参数为
  student\_name(string),
  student\_school(enum),school只能从五个学校的代号里选取。
\item
  获取所有的特色活动的信息 GET 方法到 /xingyu/get\_special\_activity.php
  没有参数 返回的结果 result{[}`special\_activity\_list'{]} 是一个array,
  每个array的 item 长度为3,依次为 name, location, time。
\item
  删除某个活动的学生 POST 方法到 /xingyu/remote\_activity\_student.php
  (JSON格式数据)必有参数为 深大的周数(week),int;{[}3-18{]}
  和小组名称(name)(string);以及学生名字列表(student\_list:{[}`name\_1',`name\_2'{]}),其中学生名字的列表为要
  删除的同学,允许同学之前没参加过该活动,此时也不会报错。 返回结果 err
  = 5 时表示该活动不存在。
\item
  更改志愿者的组别信息 POST 方法到
  /xingyu/modify\_student\_group.php?action=add 如果action=add
  是添加一个新的组别,如果action=delete 是删除这个组别(目前没有约束)。
  (JSON格式数据)必有参数为 student\_name(string),
  group\_id(int),group\_id \textgreater{} 0.
\item
  获取五校统计信息 GET 方法到 /xingyu/get\_statistics.php 返回
  JSON格式数据
  {[}\{`school':`hit',`total\_student':23,`total\_count':45\},
  \ldots{}{]} 其中 total\_student 表示该校人数信息, total\_count
  表示该校人次信息。
\item
  删除流动志愿者 POST 方法到 /xingyu/delete\_student\_flow.php
  (JSON格式数据)必有参数为 student\_name(string),
  student\_school(enum),school只能从五个学校的代号里选取。
  注意:如果这个志愿者有参加过活动,则必须先通过其他的接口取消他参加的活动才能删除。
\item
  获取各校志愿者本学期的统计信息 GET 方法到
  /xingyu/download\_summary.php?student\_school,
  school只能从五个学校的代号里选取 返回 excel
  格式数据,如果学校未识别,返回的 body 为空。
\end{enumerate}

\hypertarget{ux767bux5f55ux76f8ux5173}{%
\subsection{登录相关}\label{ux767bux5f55ux76f8ux5173}}

登录相关:第一次登录需同时完成1和2两步,假如只完成一步也算第一次登录失败。后面登录只需完成第1步。
1. 凭证校验 GET 方法到 /xingyu/openid.php?code=abc 其中 code
是从微信开发者服务器上获得的; 返回 JSON 格式的数据,有
\{``err'':1,``msg'':``invalid code'',``result'':"``\} 或者
\{''err``:0,''msg``:''``,''result``:\{''openid``:''id``,''session\_key``:''key"\}\}
。

\begin{enumerate}
\def\labelenumi{\arabic{enumi}.}
\setcounter{enumi}{1}
\tightlist
\item
  关联志愿者信息与openid。POST 方法到 /xingyu/openid.php
  提交JSON格式的数据 \{``openid'':``id'', ``nickname'':``nickname''\}
  返回 JSON 格式的数据,\{``err'':0,``msg'':"``,''result``:''"\},其中
  err 为 0 表示操作成功。
\end{enumerate}

\hypertarget{ux6743ux9650ux76f8ux5173}{%
\subsection{权限相关}\label{ux6743ux9650ux76f8ux5173}}

数据库中 student 表 里面 openid 非空者为管理员,有相应POST接口的权限。
所有 POST 请求 openid
是必有参数,可以置空或不填,但这种情况下肯定无法进行数据库的写操作。
如果在POST请求中返回的错误码 err =
44,说明当前用户没有权限执行这个操作。

\end{document}
